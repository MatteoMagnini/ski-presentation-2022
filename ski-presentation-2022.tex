%%%%%%%%%%%%%%%%%%%%%%%%%%%%%%%%%%%%%%%%%%%%%%%%%%%%%%%%%%%%%%%%%%%%%%%%%%%%%%%%
% AMS Beamer series / Bologna FC / Template
% Andrea Omicini
% Alma Mater Studiorum - Università di Bologna
% mailto:andrea.omicini@unibo.it
%%%%%%%%%%%%%%%%%%%%%%%%%%%%%%%%%%%%%%%%%%%%%%%%%%%%%%%%%%%%%%%%%%%%%%%%%%%%%%%%
%\documentclass[handout]{beamer}\mode<handout>{\usetheme{default}}
%
\documentclass[presentation]{beamer}\mode<presentation>{\usetheme{AMSBolognaFC}}
%\documentclass[handout]{beamer}\mode<handout>{\usetheme{AMSBolognaFC}}
%%%%%%%%%%%%%%%%%%%%%%%%%%%%%%%%%%%%%%%%%%%%%%%%%%%%%%%%%%%%%%%%%%%%%%%%%%%%%%%%
\usepackage{hyperref}
\usepackage{csquotes}
\usepackage{cleveref}
\crefname{item}{item}{items}
\Crefname{Item}{Item}{Items}
\usepackage{wasysym}
\usepackage[ddmmyyyy]{datetime}
\renewcommand{\dateseparator}{-}
\def\BibTeX{{\rm B\kern-.05em{\sc i\kern-.025em b}\kern-.08em
    T\kern-.1667em\lower.7ex\hbox{E}\kern-.125emX}}
\renewcommand{\thefootnote}{\fnsymbol{footnote}}
% version
\newcommand{\templatemajor}{1}
\newcommand{\templateminor}{4}
\newcommand{\templatepatch}{20220217}
\newcommand{\templateversion}{\templatemajor.\templateminor.\templatepatch}
%%%%%%%%%%%%%%%%%%%%%%%%%%%%%%%%%%%%%%%%%%%%%%%%%%%%%%%%%%%%%%%%%%%%%%%%%%%%%%%%
\title[SKI: Symbolic Knowledge Injection]
{SKI: Symbolic Knowledge Injection}
%
\subtitle[state of the art and our current works]
{state of the art and our current works}
%
\author[\sspeaker{Magnini}]
{\speaker{Matteo Magnini}\\\href{mailto:matteo.magnini@unibo.it}{matteo.magnini@unibo.it}}
%
\institute[DISI, Univ.\ Bologna]
{Dipartimento di Informatica -- Scienza e Ingegneria (DISI)\\\textsc{Alma Mater Studiorum} -- Universit{\`a} di Bologna}
%
\date[\today]{\today}
%
%%%%%%%%%%%%%%%%%%%%%%%%%%%%%%%%%%%%%%%%%%%%%%%%%%%%%%%%%%%%%%%%%%%%%%%%%%%%%%%%
\begin{document}
%%%%%%%%%%%%%%%%%%%%%%%%%%%%%%%%%%%%%%%%%%%%%%%%%%%%%%%%%%%%%%%%%%%%%%%%%%%%%%%%

%/////////
\frame{\titlepage}
%/////////

%%===============================================================================
%\section*{Outline}
%%===============================================================================
%
%%/////////
%\frame[c]{\tableofcontents[hideallsubsections]}
%%/////////

%===============================================================================
\section{Premises}
%===============================================================================

%/////////
\begin{frame}[c]{Definition}
%
We define symbolic knowledge injection as:
%
\begin{displayquote}\itshape
    any \emph{algorithmic} procedure affecting how \textcolor{bolognafcred}{sub-symbolic predictors} draw their inferences in such a way that predictions are either \emph{computed} as a function of, or made \emph{consistent} with, some \emph{given} \textcolor{bolognafcred}{symbolic knowledge}.
\end{displayquote}
%
\end{frame}
%/////////

%/////////
\begin{frame}[c]{Symbolic Knowledge}
    %
    A symbolic representation consists of:
    %
    \begin{enumerate}
        \item a set of symbols;
        \item\label{item:symbolic-combination} a set of grammatical rules governing the combining of symbols; 
        \item\label{item:symbolic-assignment} elementary symbols and any admissible combination of them can be assigned with meaning.
        %
        \begin{itemize}
            \item[$\Rightarrow$] Symbolic knowledge is both human and machine interpretable;
            \item First order logic (FOL) is an example of symbolic representation.
            %
        \end{itemize}
    \end{enumerate}
    
\end{frame}
%/////////

%/////////
\begin{frame}[c]{Sub-symbolic data}
    \begin{itemize}
        \item ML methods, and sub-symbolic approaches in general, represent data as arrays of real numbers, and knowledge as functions over such data.
        %
        \item Despite numbers are technically symbols as well, we cannot consider arrays and their functions as symbolic knowledge representation (KR) means.
        %
        \item Sub-symbolic approaches frequently violate \Cref{item:symbolic-combination,item:symbolic-assignment}.
        %
    \end{itemize}
    %
\end{frame}
%/////////

%/////////
\begin{frame}[c]{Sub-symbolic predictors}
    %
    \begin{itemize}
        \item deep neural networks (DNN);
        \begin{itemize}
            \item Convolutional neural networks (CNN);
            \item Recurrent neural networks (RNN).
        \end{itemize}
        \item kernel machines;
        \item others.
    \end{itemize}
    %
    \vfill
    %
    The vast majority of predictors are NN most probably because they are easy to manipulate and they have top performances.
\end{frame}
%/////////

%/////////
\begin{frame}[c]{Why SKI}
    %
    There are several benefits:
    %
\begin{itemize}
    \item reduce learning time;
    %
    \item reduce the data size needed for training;
    %
    \item improve predictor's accuracy;
    %
    \item build a predictor that behave as a logic engine.
\end{itemize}
%
\end{frame}
%/////////



%===============================================================================
\section{Taxonomy}
%===============================================================================

%/////////
\begin{frame}[c]{Aim}
    %
    %
\end{frame}
%/////////

%/////////
\begin{frame}[c]{Predictors}
    %
    %
\end{frame}
%/////////

%/////////
\begin{frame}[c]{How}
    %
    There exist three major ways to perform knowledge injection on sub-symbolic predictors:
    %
    \begin{itemize}
        \item constraining, a cost factor proportional to the violation of the knowledge is introduced during learning;
        \item structuring, the architecture of the predictor is built in such a way to mimic the knowledge;
        \item embedding, the symbolic knowledge is embedded into a tensor form and it is given in input as training data to the predictor.
   \end{itemize} 
    %
\end{frame}
%/////////

%/////////
\begin{frame}[c]{Constraining}
    %
    %
\end{frame}
%/////////

%/////////
\begin{frame}[c]{Structuring}
    %
    %
\end{frame}
%/////////

%/////////
\begin{frame}[c]{Embedding}
    %
    %
\end{frame}
%/////////

%/////////
\begin{frame}[c]{Logics}
    %
    %
\end{frame}
%/////////





%===============================================================================
\section{Use}
%===============================================================================

%/////////
\begin{frame}[c,allowframebreaks]{Files}
%
\begin{block}{Style files}
The files
\begin{itemize}
	\item \texttt{beamercolorthemebolognafc.sty}
	\item \texttt{beamerthemeAMSBolognaFC.sty}
	\item \texttt{almacesena-background.pdf}
\end{itemize}
should be placed either in the local folder with the main \texttt{.tex} file, or, in your Beamer system directory, e.g.
\begin{itemize}
	\item \texttt{/Users/\{username\}/Library/texmf/tex/latex/local/beamer/}
\end{itemize}
\end{block}
%
\begin{block}{BST files}
The files
\begin{itemize}
	\item \texttt{apalike-AMS.bst}
\end{itemize}
should be placed either in the local folder with the main \texttt{.tex} file, or, in your Beamer system directory, e.g.
\begin{itemize}
	\item \texttt{/Users/\{username\}/Library/texmf/bibtex/bst/local/}
\end{itemize}
\end{block}
%
\end{frame}
%/////////

%/////////
\begin{frame}[c,fragile]{Declaration}
%
\begin{block}{\texttt{\textbackslash{}documentclass}}
Your main Beamer \texttt{.tex} file should open with the declaration
%
\begin{verbatim}
    \documentclass[presentation]{beamer}
        \mode<presentation>{\usetheme{AMSBolognaFC}}
\end{verbatim}
%
so as to use the AMS Bologna FC Beamer style 
\end{block}
%
\end{frame}
%/////////

%/////////
\begin{frame}[c,fragile]{Bibliography Style}
%
\begin{block}{\texttt{apalike-AMS}}
Your main Beamer \texttt{.tex} file should include the declaration
\begin{verbatim}
    \bibliographystyle{apalike-AMS}
\end{verbatim}	
so as to use the AMS Bologna FC \BibTeX{} style 
\end{block}
%
\end{frame}
%/////////

%/////////
\begin{frame}[c,fragile]{Template}
%
\begin{block}{\texttt{AMSBolognaFC-template.tex}}
This template's sources can be used as a simple example of how yo use this Beamer style
\end{block}
%
\end{frame}
%/////////


%===============================================================================
\section{Style}
%===============================================================================

%/////////
\begin{frame}[c,allowframebreaks]{Colours for AMS Bologna FC}
%
\begin{exampleblock}{\texttt{\textcolor{bolognafcblue}{bolognafcblue}}}
	\begin{description}
		\item[HEX] \texttt{\#1A2F48}
		\item[RGB] \texttt{26,47,72}
	\end{description}
\end{exampleblock}
\begin{exampleblock}{\texttt{\textcolor{bolognafcred}{bolognafcred}}}
	\begin{description}
		\item[HEX] \texttt{\#A21C26}
		\item[RGB] \texttt{162,28,38}
	\end{description}
\end{exampleblock}
\framebreak
\begin{block}{\texttt{\textcolor{bolognafcwhite}{bolognafcwhite}}}
	\begin{description}
		\item[HEX] \texttt{\#FFFFFF}
		\item[RGB] \texttt{255,255,255}
	\end{description}
\end{block}
\begin{alertblock}{\texttt{\textcolor{bolognafcsilver}{bolognafcsilver}}}
	\begin{description}
		\item[HEX] \texttt{\#ECECEC}
		\item[RGB] \texttt{236,236,236}
	\end{description}
\end{alertblock}
%
\end{frame}
%/////////

%/////////
\begin{frame}[c,fragile]{Blocks}
%
\begin{block}{This is a \texttt{block} environment}
\begin{verbatim}
\begin{block}
...
\end{block}
\end{verbatim}
\end{block}
%
\begin{exampleblock}{This is an \texttt{exampleblock} environment}
\begin{verbatim}
\begin{exampleblock}
...
\end{exampleblock}
\end{verbatim}
\end{exampleblock}
%
\begin{alertblock}{This is an \texttt{alertblock} environment}
\begin{verbatim}
\begin{alertblock}
...
\end{alertblock}
\end{verbatim}
\end{alertblock}
%
\end{frame}
%/////////

%===============================================================================
\section{New Commands}
%===============================================================================

%/////////
\begin{frame}[c,allowframebreaks,fragile]{Citations}
%
\begin{alertblock}{\texttt{\textbackslash{}ccite} command---e.g., \ccite{bibtex-patashnik88}}
\begin{verbatim}
\ccite{bibtex-patashnik88}
\end{verbatim}
\begin{itemize}
	\item to be used instead of standard \texttt{\textbackslash{}cite} command
	\item prints as \ccite{bibtex-patashnik88}
	\item can be used as a note\ccite{bibtex-patashnik88}, with no space before
\end{itemize}
\end{alertblock}
%
\begin{exampleblock}{\texttt{\textbackslash{}cccite} command---e.g., \cccite{bibtex-patashnik88}}
\begin{verbatim}
\cccite{bibtex-patashnik88}
\end{verbatim}
\begin{itemize}
	\item a lighter version of the \texttt{\textbackslash{}ccite} command over non-dark, non-light backgrounds
	\begin{itemize}
		\item as here above in \texttt{examplebox} header
	\end{itemize}
	\item can be used as a note with no space before, in the same way as \texttt{\textbackslash{}ccite}
\end{itemize}
\end{exampleblock}
%
\end{frame}
%/////////

%/////////
\begin{frame}[c,fragile]{URLs}
%
\begin{block}{\texttt{\textbackslash{}uurl} command}
\begin{itemize}
	\item to be used instead of standard \texttt{\textbackslash{}url} command
	\item[e.g.] \verb|\uurl{http://apice.unibo.it}| prints as \uurl{http://apice.unibo.it}
\end{itemize}
\end{block}
%
\begin{block}{\texttt{\textbackslash{}uuurl} command---e.g., \uuurl{http://apice.unibo.it}}
\begin{itemize}
	\item to be used instead of standard \texttt{\textbackslash{}url} command over dark backgrounds
	\item[e.g.] see \verb|\uuurl{http://apice.unibo.it}| above in this \texttt{block} header
\end{itemize}
\end{block}
%
\end{frame}
%/////////

%/////////
\begin{frame}[c,fragile]{Alert}
%
\begin{block}{\texttt{\textbackslash{}aalert} command---e.g., \aalert{alerted text}}
\begin{itemize}
	\item to be used instead of standard \texttt{\textbackslash{}alert} command over dark backgrounds
	\item[e.g.] see \verb|\aalert{alerted text}| above in this \texttt{block} header
\end{itemize}
\end{block}
%
\end{frame}
%/////////

%/////////
\begin{frame}[c,fragile,allowframebreaks]{Speaker(s) \emph{vs.} Authors}
%
\begin{exampleblock}{\texttt{\textbackslash{}speaker} command--e.g., \speaker{Diego Zorro}}
\begin{itemize}
	\item to be used within \texttt{\textbackslash{}author} standard \BibTeX{} command to single out the actual speaker among the authors
	\item[e.g.] as in
\begin{verbatim}
	\author[Garcia \and Zorro]
	{Sarg Garcia \and \speaker{Diego Zorro}}
\end{verbatim}
	\item and in the author specification of this template
\end{itemize}
\end{exampleblock}
%
\begin{block}{\texttt{\textbackslash{}sspeaker} command--e.g., \sspeaker{Diego Zorro}}
\begin{itemize}
	\item to be used within \texttt{\textbackslash{}author} standard \BibTeX{} command to single out the actual speaker among the authors in the short form
	\item[e.g.] as in
\begin{verbatim}
	\author[Garcia \and \sspeaker{Zorro}]
	{Sarg Garcia \and \speaker{Diego Zorro}}
\end{verbatim}
	\item and in the author specification of this template
\end{itemize}
\end{block}

\end{frame}
%/////////


%===============================================================================
\section*{}
%===============================================================================

%/////////
\frame{\titlepage}
%/////////

%===============================================================================
\section*{\refname}
%===============================================================================

%%%%
\setbeamertemplate{page number in head/foot}{}
%/////////
\begin{frame}[c,noframenumbering]{\refname}
%\begin{frame}[t,allowframebreaks,noframenumbering]{\refname}
%	\tiny
	\scriptsize
%	\footnotesize
	\bibliographystyle{apalike-AMS}
	\bibliography{AMSBolognaFC-template}
\end{frame}
%/////////

%%%%%%%%%%%%%%%%%%%%%%%%%%%%%%%%%%%%%%%%%%%%%%%%%%%%%%%%%%%%%%%%%%%%%%%%%%%%%%%%
\end{document}
%%%%%%%%%%%%%%%%%%%%%%%%%%%%%%%%%%%%%%%%%%%%%%%%%%%%%%%%%%%%%%%%%%%%%%%%%%%%%%%%
